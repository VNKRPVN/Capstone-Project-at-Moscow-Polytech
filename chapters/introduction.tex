\newpage
\begin{center}
  \textbf{\large АННОТАЦИЯ}
\end{center}


Наименование работы: Модуль управления сессиями пользователей в Keycloak.

Цель работы: Разработать расширение для Keycloak, позволяющее вести детализированный учет активных сессий пользователей, обнаруживать аномальные входы, автоматически завершать неактивные сессии и предоставлять администраторам средства аналитики и контроля за активностью пользователей.

Объект исследования: механизм управления сессиями в системе аутентификации и авторизации Keycloak.
Предмет исследования: методы мониторинга и анализа пользовательских сессий, алгоритмы обнаружения подозрительной активности, автоматизированные процессы завершения сессий.

\onehalfspacing
\setcounter{page}{2}

\newpage
\renewcommand{\contentsname}{\centerline{\large СОДЕРЖАНИЕ}}
\tableofcontents

\newpage
\begin{center}
  \textbf{\large ВВЕДЕНИЕ}
\end{center}
\addcontentsline{toc}{chapter}{ВВЕДЕНИЕ}


\textbf{Актуальность}

В современных информационных системах управление аутентификацией и авторизацией играет ключевую роль в обеспечении безопасности и удобства доступа пользователей. Одним из наиболее популярных решений для организации единого входа (Single Sign-On, SSO) и управления доступом является Keycloak – мощная и гибкая система, предоставляющая возможности централизованной идентификации и аутентификации пользователей. Этот инструмент широко применяется в корпоративных системах, облачных платформах и микросервисных архитектурах, поскольку позволяет администраторам управлять учетными записями, настраивать механизмы многофакторной аутентификации, интегрироваться с различными внешними провайдерами идентификации и обеспечивать контроль над сессиями пользователей.

Основное преимущество Keycloak заключается в его универсальности и возможности бесшовной интеграции с существующими системами. Он поддерживает распространенные протоколы аутентификации, такие как OAuth 2.0, OpenID Connect и SAML, что делает его удобным решением для организаций, использующих множество различных сервисов. Администраторы могут централизованно управлять доступами пользователей, устанавливать политики безопасности и настраивать права в зависимости от ролей. Гибкость Keycloak позволяет легко масштабировать систему, обеспечивая надежность и отказоустойчивость при работе с большими объемами данных и запросов.

Несмотря на свою мощность и функциональность, Keycloak обладает рядом недостатков, которые ограничивают его возможности в управлении сессиями и мониторинге активности пользователей. Одним из ключевых недостатков является отсутствие встроенных механизмов детального анализа сессий и выявления аномальной активности. В базовой версии системы отсутствуют инструменты для отслеживания подозрительных входов, автоматического завершения неактивных сессий и визуального анализа поведения пользователей. Администраторы могут лишь просматривать активные сессии и вручную завершать их при необходимости, но не имеют средств для выявления потенциальных угроз, таких как входы с незарегистрированных устройств, попытки взлома путем подбора пароля или логины с разных географических локаций в короткие промежутки времени. В условиях растущих киберугроз и повышенных требований к безопасности такие ограничения становятся серьезной проблемой для организаций, использующих Keycloak в качестве основного механизма аутентификации.

Разрабатываемое расширение призвано устранить указанные недостатки и значительно улучшить функциональность Keycloak в области управления сессиями пользователей. Оно позволит не только вести детализированный учет сессий, но и анализировать активность пользователей, автоматически завершать сессии при обнаружении аномалий и интегрироваться с внешними сервисами для определения геолокации входящих запросов. Внедрение механизма обнаружения аномальной активности обеспечит защиту от несанкционированных входов, предотвращая использование скомпрометированных учетных данных и снижая риск атак на систему. Администраторы получат удобный интерфейс для мониторинга сессий, в котором будут отображаться все активные подключения, их продолжительность, местоположение пользователей и уровень риска каждой сессии. Кроме того, система сможет автоматически завершать неактивные или подозрительные сессии, предотвращая их возможное использование злоумышленниками.

Разработка данного расширения сделает Keycloak более эффективным инструментом для защиты учетных записей и управления доступами. Оно позволит не только повысить уровень безопасности аутентификации, но и упростить администрирование сессий, сократив риски, связанные с компрометацией учетных данных. В условиях современных киберугроз важно не только обеспечивать корректную аутентификацию пользователей, но и контролировать их активность в системе, выявляя потенциальные попытки взлома и своевременно реагируя на них. Реализуемые в рамках расширения механизмы мониторинга, анализа и автоматического завершения сессий помогут организациям значительно повысить защищенность своих систем, минимизируя риски утечки данных и несанкционированного доступа.

\textbf{Аналоги Keycloak}

Одним из основных конкурентов Keycloak является Okta – облачное решение, ориентированное на корпоративный сегмент. Оно предлагает мощную систему аутентификации, интеграцию с множеством сторонних сервисов и удобные инструменты управления пользователями. Okta широко применяется в крупных компаниях благодаря своей масштабируемости, гибким настройкам политик безопасности и поддержке многофакторной аутентификации. Однако ключевым недостатком этой системы является закрытый исходный код и модель лицензирования, что делает её менее доступной для организаций, стремящихся к использованию решений с открытым кодом. В отличие от Keycloak, Okta требует подписки на платные тарифные планы, что может стать критическим фактором при выборе системы аутентификации.

Другим популярным решением является Auth0, которое также ориентировано на облачную идентификацию и предоставляет широкий спектр возможностей для интеграции с различными платформами. Auth0 обладает удобным интерфейсом, поддерживает кастомные сценарии аутентификации и адаптирован для работы с веб-приложениями и мобильными устройствами. Однако, как и Okta, Auth0 является коммерческим продуктом, что накладывает определённые ограничения на его использование в корпоративных системах с большими объемами пользователей. Кроме того, в отличие от Keycloak, который может быть развернут на собственной инфраструктуре, Auth0 требует использования облачных сервисов, что не всегда отвечает требованиям безопасности организаций с высокими стандартами защиты данных.

Одним из мощных решений в области управления аутентификацией является IBM Security Access Manager (ISAM), предлагающий гибкие механизмы контроля доступа, единого входа и многофакторной аутентификации. Данный продукт ориентирован на крупные предприятия, обеспечивая высокую степень защиты и масштабируемость. Однако его сложность в настройке и интеграции, а также высокая стоимость лицензирования делают его менее привлекательным для средних и небольших компаний. В отличие от Keycloak, который может быть развернут в контейнеризированной среде с минимальными затратами, ISAM требует значительных ресурсов для внедрения и поддержки.

Microsoft Entra ID (ранее Azure Active Directory) также выступает в качестве серьезного конкурента Keycloak, особенно в корпоративных средах, использующих экосистему Microsoft. Оно предоставляет продвинутые механизмы контроля доступа, включая ролевое управление, интеграцию с Windows Server Active Directory и поддержку множества аутентификационных протоколов. Однако его основное ограничение заключается в зависимости от облачной инфраструктуры Microsoft, что делает его менее удобным для гибридных и частных облачных сред. В отличие от Keycloak, который поддерживает развертывание в любых средах, включая on-premises и частные облака, Microsoft Entra ID требует привязки к сервисам Microsoft, что может ограничивать его использование в гетерогенных инфраструктурах.

Ещё одним решением, заслуживающим внимания, является WSO2 Identity Server – система управления идентификацией с открытым кодом, предлагающая поддержку OAuth 2.0, OpenID Connect и SAML. WSO2 Identity Server имеет схожую с Keycloak архитектуру и позволяет централизованно управлять учетными записями, сессиями и политиками безопасности. Однако его основное отличие заключается в ориентации на сложные сценарии интеграции с микросервисными архитектурами, что делает его более сложным в настройке по сравнению с Keycloak. Кроме того, Keycloak выигрывает за счёт активного сообщества разработчиков и лучшей документации, что облегчает его использование и поддержку.

пункты:
\begin{enumerate}
\item пункт 1;
\item пункт 2.
\end{enumerate}

\newpage

\textbf{Цель бакалаврской квалификационной работы} -- цели.

\textbf{Задачи бакалаврской квалификационной работы:}
\begin{enumerate}
\item Задача 1. 
\item Задача 2.
\item Задача 3.
\item Задача 4. 
\end{enumerate}

